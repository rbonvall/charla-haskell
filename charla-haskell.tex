\documentclass[12pt]{beamer}
\usepackage[spanish]{babel}
\usepackage[utf8]{inputenc}
\usepackage{xcolor}
\usepackage{listings}

\usecolortheme{crane}

\usepackage{palatino}
\usefonttheme{serif}

\title {Programación funcional en Haskell}
\author{Roberto Bonvallet}
\institute[]{
    Departamento de Informática \\
    Universidad Técnica Federico Santa María
}
\date{Mayo de 2009}

\lstloadlanguages{c,haskell}
\lstdefinestyle{hs}{
    language=haskell,
    literate={->}{{$\rightarrow$}}1
             {<-}{{$\leftarrow$}}1
             {=>}{{$\Rightarrow$}}1
             {>=}{{$\ge$}}1
             {<=}{{$\le$}}1
             {'}{{}}0
             {\ .\ }{{~$\circ$~}}3
             {\\}{{$\lambda$}}1,
}
\lstdefinestyle{c}{
    language=c++,
    rangeprefix={/**},
    rangesuffix={**/},
}

\begin{document}

\begin{frame}
    \titlepage
\end{frame}

\begin{frame}
    \begin{block}{Programas y diapositivas}
        %git~clone~\url{git://github.com/rbonvall/charla-haskell.git}
        git~~clone~~git:/\!\!/github.com/rbonvall/charla-haskell.git
    \end{block}
\end{frame}
    
\begin{frame}[fragile]
    \lstinputlisting[style=c]{programas/01-struct.c}
\end{frame}

\begin{frame}[fragile]
    \lstinputlisting[style=c]{programas/02-struct-init.c}
\end{frame}

\begin{frame}[fragile]
    \lstinputlisting[style=c]{programas/03-struct-maxint.c}
\end{frame}

\begin{frame}[fragile]
    \lstinputlisting[style=hs]{programas/qs.hs}
\end{frame}

\begin{frame}[fragile]
    \begin{block}{Quicksort}
        \lstinputlisting[language=Haskell]{programas/qs.hs}
    \end{block}
\end{frame}

\begin{frame}[fragile]
    \begin{block}{¿Qué hace esto?}
        \lstinputlisting[style=c,linerange=11-13]{programas/reduce.cpp}
    \end{block}
    \begin{block}{¿Y esto?}
        \lstinputlisting[style=c,linerange=21-23]{programas/reduce.cpp}
    \end{block}
    %}
\end{frame}


\begin{frame}
    \frametitle{Recursos}
    \begin{thebibliography}{99}
        \bibitem{HaskellWiki}
        Haskell Wiki
        \newblock \url{http://www.haskell.org/haskellwiki/}

        \bibitem{LearnYouAHaskell}
        Learn You a Haskell For Great Good!
        \newblock \url{http://learnyouahaskell.com/}

        \bibitem{RealWorldHaskell}
        Real World Haskell
        \newblock \url{http://book.realworldhaskell.org/read/}
    \end{thebibliography}
\end{frame}

\end{document}

