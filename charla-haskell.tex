\documentclass[12pt]{beamer}
\usepackage[spanish]{babel}
\usepackage[utf8]{inputenc}
\usepackage{xcolor}
\usepackage{listings}
\usepackage{array}

\usecolortheme{crane}

\usepackage{palatino}
\usefonttheme{serif}

\usepackage{hyperref}

\title {Programación funcional en Haskell}
\author{Roberto Bonvallet}
\institute[]{
    Departamento de Informática \\
    Universidad Técnica Federico Santa María
}
\date{Mayo de 2009}

\lstloadlanguages{c,haskell}
\lstdefinestyle{hs}{
    language=haskell,
    literate={->}{{$\rightarrow$}}1
             {<-}{{$\leftarrow$}}1
             {=>}{{$\Rightarrow$}}1
             {>=}{{$\ge$}}1
             {<=}{{$\le$}}1
             %{'}{{}}0
             {\ .\ }{{~$\circ$~}}3
             {\\}{{$\lambda$}}1,
}
\lstdefinestyle{c}{
    language=c++,
    rangeprefix={/**},
    rangesuffix={**/},
}

\begin{document}

\begin{frame}
    \titlepage
\end{frame}

\begin{frame}
    \begin{block}{Programas y diapositivas}
        %git~clone~\url{git://github.com/rbonvall/charla-haskell.git}
        git~~clone~~git:/\!\!/github.com/rbonvall/charla-haskell.git
    \end{block}
\end{frame}
    
%\begin{frame}[fragile]
%    \lstinputlisting[style=c]{programas/01-struct.c}
%\end{frame}
%
%\begin{frame}[fragile]
%    \lstinputlisting[style=c]{programas/02-struct-init.c}
%\end{frame}
%
%\begin{frame}[fragile]
%    \lstinputlisting[style=c]{programas/03-struct-maxint.c}
%\end{frame}

\begin{frame}[fragile]
    \begin{block}{¿Qué hace esto?}
        \lstinputlisting[style=c,linerange=13-15]{programas/fold.cpp}
    \end{block}
    \pause
    \begin{block}{¿Y esto?}
        \lstinputlisting[style=c,linerange=21-23]{programas/fold.cpp}
    \end{block}
\end{frame}

\begin{frame}[fragile]
    \begin{block}{¿Y esto otro?}
        \lstinputlisting[style=c,linerange=29-31]{programas/fold.cpp}
    \end{block}
    \pause
    \begin{block}{¿Y esto de ac\'a?}
        \lstinputlisting[style=c,linerange=37-39]{programas/fold.cpp}
    \end{block}
\end{frame}

\begin{frame}[fragile]
    \begin{block}{El patr\'on com\'un}
        \LARGE
        \begin{lstlisting}[style=c,escapechar=\&,gobble=12]
            z = &\alert{z$_0$}&;
            for (i = 0; i < n; ++i)
                z = &\alert{f(}&z, a[i]&\alert{)}&;
        \end{lstlisting}
    \end{block}
\end{frame}

\begin{frame}[fragile]
    \begin{block}{En ``funcional''}
        \lstinputlisting[style=hs,linerange=1-12]{programas/fold.hs}
    \end{block}
\end{frame}

\begin{frame}[fragile]
    \begin{block}{Función \lstinline[style=hs]!foldl!}
        \lstinputlisting[style=hs,linerange=15-19]{programas/fold.hs}
    \end{block}
\end{frame}

\begin{frame}[fragile]
    \begin{block}{Definición de \lstinline[style=hs]!foldl!}
        \lstinputlisting[style=hs,linerange=22-24]{programas/fold.hs}
    \end{block}
\end{frame}

\begin{frame}[fragile]
    \begin{block}{Definición de funciones}
        \begin{lstlisting}[style=hs,gobble=12]
            fact 0 = 1
            fact x = x * fact (x - 1)
        \end{lstlisting}
    \end{block}
    \pause
    \begin{block}{Operadores son funciones}
        \begin{lstlisting}[style=hs,gobble=12]
            41 * 200
            (*) 41 200
        \end{lstlisting}
    \end{block}
    \pause
    \begin{block}{Funciones son operadores}
        \begin{lstlisting}[style=hs,gobble=12]
            elem 2 [1..10]
            2 `elem` [1..10]
        \end{lstlisting}
    \end{block}
\end{frame}

\begin{frame}[fragile]
    \begin{block}{Quicksort}
        \Large
        \vspace{1ex}
        \centering
        \begin{tabular}{*{8}{ >{\hfil} p{1em} <{\hfil} }} \hline
            \multicolumn{3}{|c}{} & $p$ & \multicolumn{4}{c|}{}              \\\hline
            &&&&&&& \\\cline{1-4}\cline{6-8}
            \multicolumn{4}{|c|}{$<p$} & $p$ & \multicolumn{3}{|c|}{$\ge p$} \\\cline{1-4}\cline{6-8}
        \end{tabular}
    \end{block}
    \pause
    \begin{block}{En Haskell}
        \lstinputlisting[style=hs]{programas/qs.hs}
    \end{block}
\end{frame}


\begin{frame}[fragile]
    \begin{block}{Dígito verificador como si estuviera en primero}
        \Large
        \vspace{1ex}
        \begin{tabular}{*{9} {>{\hfil} p{1em} <{\hfil} }}
                       & 1 & 4 & 2 & 3 & 0 & 1 & 2 & 4  \\
            \only<-1>{ &   &   &   &   &   &   &   &   } 
            \only<2->{\only<3->{$\times$}
                       & 3 & 2 & 7 & 6 & 5 & 4 & 3 & 2 } \\ %\only<3->{\cline{2-9}}
            \only<-2>{ &   &   &   &   &   &   &   &   } 
            \only<3->{ & 3 & 8 &14 &18 & 0 & 4 & 6 & 8 }
        \end{tabular}
        \begin{tabular}{c}
            $\quad$
            \only<4->{ $\xrightarrow{\quad+\quad} 61$ }
            \only<5->{ $\xrightarrow{11 - x}    -50 $ }
            \only<6->{ $\xrightarrow{\mod 11\;}   5 $ }
        \end{tabular}
    \end{block}
\end{frame}

\begin{frame}[fragile]
    \begin{block}{Fibonacci perezoso}
        \begin{lstlisting}[style=hs,gobble=12]
            fib = 1 : 1 : (zipWith (+) fib $ tail fib)
        \end{lstlisting}
    \end{block}
\end{frame}

\begin{frame}
    \frametitle{Recursos}
    \begin{thebibliography}{99}
        \bibitem{HaskellWiki}
        Haskell Wiki
        \newblock \url{http://www.haskell.org/haskellwiki/}

        \bibitem{LearnYouAHaskell}
        Learn You a Haskell For Great Good!
        \newblock \url{http://learnyouahaskell.com/}

        \bibitem{RealWorldHaskell}
        Real World Haskell
        \newblock \url{http://book.realworldhaskell.org/read/}

        \bibitem{HaskellCafe}
        Haskell Cafe
        \newblock \url{http://news.gmane.org/gmane.comp.lang.haskell.cafe}
    \end{thebibliography}
\end{frame}

\end{document}

